
\chapter{Dataset Construction}
\phantomsection
\label{ch:dataset}

\section{Text Source Collection}
\label{sec:text-source}
This section describes the process of collecting Khmer text data from various sources to create a comprehensive dataset for OCR training and evaluation.

\subsection{Khmer Websites and Dictionaries}
\label{subsec:websites}
We gathered text samples from popular Khmer news websites, online dictionaries, and digital libraries to ensure diverse vocabulary coverage and writing styles.

\subsection{Online NLP Resources and Tools} 
\label{subsec:nlp-tools}
Additional text data was collected using available Khmer NLP tools and resources, including pre-existing corpora and language processing utilities.

\section{Text Cleaning and Preprocessing}
\label{sec:preprocessing}
Raw text data underwent several preprocessing steps to ensure quality and consistency for synthetic image generation.

\subsection{Removal of Invalid Characters and Whitespace}
\label{subsec:cleaning}
We implemented filtering mechanisms to remove invalid Unicode characters, normalize whitespace, and handle special characters that could affect OCR performance.

\subsection{Unicode Normalization}
\label{subsec:unicode}
All text was normalized to ensure consistent Unicode representation of Khmer characters and their combinations.

\section{Sentence Segmentation and Reconstruction}
\label{sec:segmentation}
The cleaned text was processed to create meaningful sentence units suitable for OCR training.

\subsection{Tokenization Using khmer-nltk}
\label{subsec:tokenization}
We utilized the khmer-nltk library to perform accurate tokenization of Khmer text while preserving linguistic properties.

\subsection{Sentence Length Variation}
\label{subsec:length}
Sentences were segmented and reconstructed to create samples with varying lengths, ensuring the dataset represents real-world text diversity.

\section{Image Generation Pipeline}
\label{sec:generation}
A robust pipeline was developed to convert processed text into synthetic training images.

\subsection{Font and Background Selection}
\label{subsec:fonts}
Multiple Khmer fonts and background variations were incorporated to create diverse and realistic training samples.

\subsection{Noise Injection Techniques}
\label{subsec:noise}
Various types of noise and distortions were systematically added to simulate real-world document conditions.

\subsection{Image Rotation and Margin Augmentation}
\label{subsec:augmentation}
Geometric transformations and margin variations were applied to improve model robustness to different text orientations and layouts.

\section{Dataset Statistics and Format}
\label{sec:statistics}
This section presents detailed statistics about the generated dataset, including size, character distribution, and format specifications.

\section{Comparison with Existing Datasets}
\label{sec:comparison}
A comparative analysis of our dataset with existing Khmer OCR datasets, highlighting improvements and unique characteristics.
