\chapter{Discussion}
\phantomsection
\label{ch:discussion}

\section{Effectiveness of Synthetic Data}
\label{sec:effectiveness}
The effectiveness of synthetic data in training our OCR system has been demonstrated 
through several key findings. Our experiments showed that synthetic data generation 
significantly improved the model's performance, particularly in handling diverse 
font styles and text layouts. The model trained on synthetic data achieved a 
Character Error Rate (CER) of 0.05 and Word Error Rate (WER) of 0.03, which is 
comparable to state-of-the-art results in similar OCR tasks.

The synthetic data generation approach proved particularly valuable for Khmer text 
recognition, where the availability of real-world training data is limited. 
By generating synthetic samples with controlled variations in font styles, sizes, 
and text arrangements, we were able to create a diverse training dataset that helped 
the model learn robust features for text recognition. This is evidenced by the model's 
ability to generalize to approximately 70 different font styles despite being trained 
on only 15 different fonts.

However, our analysis also revealed some limitations in the synthetic data approach. 
The model showed reduced performance when dealing with highly curved or circular text 
arrangements, as well as with artistic text styles that deviate significantly from standard fonts. 
This suggests that while synthetic data is effective for training basic text recognition 
capabilities, it may not fully capture the complexity and variety of real-world text appearances.

The success of our synthetic data approach highlights its potential as a viable solution 
for low-resource language OCR systems. This finding is particularly relevant for other 
languages with limited training data availability, suggesting that similar approaches 
could be applied to improve OCR systems for other low-resource languages.

\section{Strengths and Limitations of the OCR System}
\label{sec:strengths} 
A critical examination of the system's capabilities and areas for improvement, based on experimental results.

\section{Research Challenges and Lessons Learned}
\label{sec:challenges}
Discussion of key technical and methodological challenges encountered during the research, and important lessons learned.

\section{Comparison with Related Works}
\label{sec:related-works}
Analysis of how our approach and results compare with other recent work in Khmer OCR and related low-resource language OCR systems.

\section{Impact on Khmer NLP and OCR Research}
\label{sec:impact}
Discussion of the broader implications of this work for Khmer language technology and OCR research in general.
