\documentclass[12pt]{report}
\usepackage{fontspec}
\usepackage{graphicx}
\usepackage{geometry}
\usepackage{fancyhdr}

\geometry{
  a4paper,
  top=1.5cm,
  bottom=2.0cm,
  left=2.0cm,
  right=2.0cm
}


% Khmer and English fonts
\newfontfamily\khmerfont{Khmer OS Muol Light}
\newfontfamily\khmernormal{Khmer OS Fasthand}
\newfontfamily\englishfont{TeX Gyre Termes}

\title{}
\author{}
\date{}

\begin{document}

% Title page with no page number
\begin{titlepage}
    \centering
    \vspace*{1cm}

    \begin{minipage}{0.25\textwidth}
        \includegraphics[width=3.5cm]{figures/RUPP.jpg}
    \end{minipage}
    \hfill
    \begin{minipage}{0.7\textwidth}
        \raggedright
        {\khmerfont\fontsize{16pt}{20pt}\selectfont សាកលវិទ្យាល័យភូមិន្ទភ្នំពេញ\\[0.6em]}
        {\large\bfseries ROYAL UNIVERSITY OF PHNOM PENH}
    \end{minipage}

    \vspace{2cm}

    {\khmerfont\fontsize{12pt}{20pt}\selectfont ការស្រាវជ្រាវវិធីសាស្ដ្រថ្មី  សម្រាប់កំណត់សម្គាល់អត្ថបទអក្សរខ្មែរ ដែលសរសេរដោយដៃ នឹងបានប្រើប្រាស់ស្ថាបត្យកម្ម Craft ជាមួយនឹង TrOCR\\[0.4em]}
    {\englishfont\fontsize{15pt}{20pt}\selectfont\bfseries A novel End-to-End approach to Khmer Handwritten Text \par}
    {\englishfont\fontsize{15pt}{20pt}\selectfont\bfseries Recognition using Craft with TrOCR Architecture \par}

    \vspace{3.0cm}

    {\englishfont\fontsize{16pt}{20pt}\selectfont Mr. Vitou Soy\par}

    \vspace{3.0cm}

    {\englishfont\fontsize{16pt}{20pt}\selectfont A Thesis\par}
    \vspace{0.5cm}
    {\large In Partial Fulfilment of the Requirement for the Degree of\par}
    {\large Bachelor of Engineering in Information-Technology-Engineering\par}
    {\large Academic Year 2022 - 2025\par}

    \vspace{2.5cm}

    {\englishfont
    \begin{center}
        \begin{tabular}{ll}
            {Examination committee:} & Mr. Sokchea Kor (Advisor) \\
                                            & Mr. Chanpiseth Chap (committee) \\
                                            & Mrs. Daly Chea (committee)\\
                                            & Dr. \dotfill
        \end{tabular}
    \end{center}
    }


    \vfill
\end{titlepage}

\thispagestyle{empty}


% --- Abstract Page (English) ---
\thispagestyle{plain}
\begin{center}
    {\khmerfont\fontsize{16pt}{20pt}\selectfont មូលន័យសង្ខេប \par}
    \vspace{1cm}
    \begin{minipage}{0.9\textwidth}
        \khmernormal
        \small
        ការស្រាវជ្រាវនេះផ្តោតលើវិធីសាស្ត្រថ្មីសម្រាប់ការទទួលស្គាល់អត្ថបទសរសេរដោយ ដៃភាសាខ្មែរ។ វាប្រើបច្ចេកវិទ្យា TrOCR ដែលបានបណ្ដុះបណ្ដាលជាមុនលើទិន្នន័យ ពហុភាសាថ្នាក់ធំ ហើយបន្តដំណើរការបណ្ដុះបណ្ដាលបន្ថែមជាផ្នែកបន្ថែម។ ទិន្នន័យសម្រាប់ការបណ្ដុះបណ្ដាលត្រូវបានរៀបចំយ៉ាងប្រុងប្រយ័ត្ន ដើម្បីអាចផ្គូផ្គងនឹងលក្ខណៈពិសេសនៃអក្សរខ្មែរ។ លទ្ធផលនៃការសាកល្បងបង្ហាញថា TrOCR អាចផ្ដល់នូវលទ្ធភាពស្គាល់អក្សរដែលមានតម្លៃខ្ពស់ជាងវិធីបុរាណ។
    \end{minipage}
\end{center}
\newpage

% --- Abstract Page (Khmer) ---
\thispagestyle{plain}
\begin{center}
    {\bfseries\LARGE Abstract \par}
    \vspace{1cm}
    \begin{minipage}{0.9\textwidth}
        \englishfont
        \small
        A novel approach to Khmer handwritten text recognition is proposed using a fine-tuned TrOCR transformer model. The system leverages large-scale multilingual pretraining and supervised fine-tuning to handle complex character structures in Khmer script. This thesis presents the dataset preparation, model architecture, training methodology, and performance evaluation using standard benchmarks. The results show significant improvements in recognition accuracy compared to traditional OCR techniques.
    \end{minipage}
\end{center}
\newpage

% Table of contents (no page number on first page)
\tableofcontents
\newpage

% Enable normal page numbering (centered at bottom)
\pagestyle{plain}

\chapter{Introduction}
\clearpage
\pagenumbering{arabic}
\setcounter{page}{1}

\phantomsection
\label{ch:intro}
\chapter{Introduction}

% This chapter presents about what is OCR? why it is needed, and
% what are the challenges in OCR. It also presents the problem 
% for OCR on Khmer script (non-latin based), and OCR on multilingual
% language such as Khmer and English. We will talk about the scope
% of this research because it'll help us to keep 
% our research scope ensures clarity, direction, and feasibility 
% throughout the study.

\section{Background to the Study}
\label{sec:background}

Optical Character Recognition (OCR) has changed how we 
turn printed text into digital formats. Thanks to AI 
advances, OCR systems now use deep learning to detect 
and classify characters from images. This technology 
powers digital libraries, search systems, and language 
processing tools.
OCR works great for major languages like English, 
Chinese, and Japanese. These languages have tons 
of training data and well-studied text structures. 
However, OCR for complex-script languages like Khmer
is still limited.
Cambodia needs OCR technology more than ever. 
Over the past 20 years, the country has gone digital 
fast. People want to digitize Khmer documents for 
education, research, and everyday use, but here's 
the problem: Khmer script is incredibly complex.
Khmer writing goes back to the 7th century. It's 
not like English where letters sit in a row. Khmer 
characters stack on top of each other. They have 
subscripts, diacritics, and vowel markers that can 
appear above, below, or around the main character. 
Miss one tiny mark and you change the whole meaning 
of a word.

\begin{table}[H]
    \caption{Why Khmer OCR is Desperately Needed}
    \vspace{10pt}
    \phantomsection
    \label{sec:textbook}
    \resizebox{\textwidth}{!}{
    \begin{tabular}{|l|l|l|}
    \hline
    Sector & Current Problem & Impact \\
    \hline
    Education & Physical textbooks only & Students can't search or edit content \\
    Libraries & Books rotting on shelves & Knowledge becomes inaccessible \\
    Government & Paper records everywhere & Slow bureaucracy, hard to find documents \\
    Healthcare & Handwritten patient files & Doctors waste time, medical errors increase \\
    Business & Manual data entry & Companies lose money on inefficiency \\
    Culture & Ancient texts deteriorating & We're losing our heritage \\
    \hline
    \end{tabular}
    }
\end{table}

Look at education. Most school textbooks exist only on paper. The original digital files? Gone. Lost. This creates real problems for students who need accessible learning materials.

But it's bigger than just schools. Ancient palm leaf manuscripts are crumbling. Government documents pile up in storage rooms. Hospitals still use paper files that doctors can't read properly. Businesses waste hours typing data that OCR could handle in minutes.

And here's what's really frustrating: while Google can read English text perfectly, it struggles with basic Khmer sentences. The technology gap is huge.

The thing is, Khmer OCR isn't just a nice to have anymore. It's essential for Cambodia's digital future. The country needs this technology to preserve its culture, modernize its institutions, and give its people better access to information.

That's exactly why this research matters. We're not just building another OCR system. We're creating technology that could unlock thousands of years of Khmer knowledge and make it searchable, editable, and accessible to everyone.


\section{Problem Statement}
\label{sec:problem}

To develop OCR for English is really hard, but the way much more harder than you think is 
OCR on mixing languages such as English and Khmer. Most OCR systems work great with English because English is simple. Letters sit 
in a line. You read left to right, and Done.

For Khmer, that's a completely different beast. Characters stack on top of each other. 
They have tiny marks above and below that change the meaning. Miss one little dot 
and you've got the wrong word entirely.

\begin{figure}[H]
    \centering
    \includegraphics[width=\textwidth]{figures/example_of_text_format.png}
    \caption{Example of Khmer text format showing the complexity of character combinations and diacritics}
    \label{fig:text_format}
\end{figure}

Here's what makes Khmer OCR so different. First, there are no clear word breaks. 
In English, they use spaces between words (it's easy for AI to detect word level). 
Khmer writers use spaces sometimes, sometimes they don't, as you can see in figure
\ref{fig:sequential_text}, so it's really inconsistencies wirting system. This 
makes it impossible to know where one word ends and another begins.

\begin{figure}[H]
    \centering
    \includegraphics[width=\textwidth]{figures/example_of_long_text.png}
    \caption{Example of sequential Khmer text showing how characters combine to form syllables and words}
    \label{fig:sequential_text}
\end{figure}

A growing challenge in modern Cambodia is the increasing prevalence of mixed-language 
documents that combine Khmer and English text. As English education and international 
business have expanded over the past two decades, it's become common to see documents, 
signs, textbooks, and digital content that seamlessly blend both languages within the 
same sentence or paragraph.

\begin{figure}[H]
    \centering
    \includegraphics[width=\textwidth]{figures/mix_language_khmer_and_english.png}
    \caption{Example of mixed Khmer-English text showing how both languages appear together in modern Cambodian documents}
    \label{fig:mix_language}
\end{figure}

This mixed-language phenomenon creates several specific problems for OCR systems:
\begin{itemize}
    \item \textbf{Script switching confusion:} OCR models must quickly adapt between completely different writing systems within the same text line
    \item \textbf{Different text layouts:} English flows horizontally while Khmer has vertical stacking, creating complex spatial relationships
    \item \textbf{Font inconsistencies:} The same document often uses different fonts for Khmer and English portions, confusing recognition model
\end{itemize}

Most existing Khmer OCR systems are designed for single-language scenarios 
and perform poorly when encountering this mixed-language reality 
that's everywhere in Cambodia today. Then there's the data problem. To develop high 
accuracy OCR model needs tons of millions annotated images. English has millions 
of labeled images already, How about Khmer lanugage? 
Finding sufficient training data presents a significant challenge. 
While English has millions of labeled training samples, Khmer language 
resources are extremely limited, with only a few thousand quality annotated 
samples available. The situation becomes even more challenging when seeking 
properly labeled mixed-language datasets that combine Khmer and English text. Without enough 
training data, OCR model stays dumb. It can't learn the 
patterns to recognize text accurately.

\begin{figure}[H]
    \centering
    \includegraphics[width=\textwidth]{figures/varianty_of_font.png}
    \caption{Examples of the same Khmer text rendered in different fonts, demonstrating the significant visual variations that OCR systems must handle}
    \label{fig:font_variants}
\end{figure}

And let's talk about fonts. English has maybe around 15 to 25 common fonts that most people use,
based on reported study use case, from website such as rigorousthemes.com, lifehack.org, and indeed.com. 
Khmer has many different fonts that look very unique from each other, some are thick and bold, others are 
thin and delicate, some have fancy decorations. To train model on one font and 
it fails completely on another.

Look at Figure \ref{fig:font_variants}. Same text, different fonts. To a human, 
it's obviously the same sentence, to a computer, it might be different look.

\begin{figure}[H]
    \centering
    \includegraphics[width=\textwidth]{figures/text_stacking_mixing_language.png}
    \caption{Illustration of Khmer text stacking patterns, 
    showing how characters combine vertically and horizontally 
    to form syllables and words \citep{buoy2023khmerocr}}
    \label{fig:text_stacking}
\end{figure}

That's exactly why we need a better approach. Current OCR tools aren't 
built for the messy reality of Khmer text. They expect perfect conditions
and worked with char-level, word-level only,
that just don't exist in the real world.

\section{Aim and Objectives of the Study}
\label{sec:objectives}

Here's what we're trying to do. We want to build an OCR system that actually 
works for real Khmer documents. Not just clean textbook pages, but the messy, 
mixed-language stuff you see everywhere in Cambodia. Our main goal is simple: 
create an OCR system that can handle both Khmer and 
English text in the same document with high accuracy. Something that doesn't 
fall apart when it sees a content on social media or a textbook page.

To get there, we need to tackle specific problems:

\begin{enumerate}
    \item \textbf{Create a text annotation tool:    } Before we can train anything, we need a way to mark up images with bounding boxes and text labels. We'll build our own tool that can handle both tasks such as text detection and text recognition efficiently.

    \item \textbf{Generate synthetic data that actually helps:} Real data is expensive and time-consuming to collect. We'll create millions of synthetic images that look realistic enough to train our models. Different fonts, various backgrounds, realistic noise and distortions.

    \item \textbf{Build an end-to-end OCR pipeline:} We're using CRAFT for text detection and TrOCR for recognition. But we need to modify them to work well with Khmer script and mixed-language content. This means customizing the models, not just using them out of the box.

    \item \textbf{Make it work in the real world:} Our system needs to handle text lines, not just individual characters. It should work on documents with varying quality, different fonts, and mixed Khmer-English content.

    \item \textbf{Beat existing solutions:} We want to achieve better accuracy than current OCR tools like Tesseract when it comes to Khmer text (excluding commercial models). And we want to do it reliable, not just for testing a few samples.
    
    \item \textbf{Provide an easy-to-use inference model:} Instead of hosting a public API, we’ll share our OCR model on Hugging Face so developers and businesses can run it themselves with minimal setup—no need to dive into technical details. Making the model accessible allows others to test its quality and helps the community continue improving Khmer OCR systems.
\end{enumerate}

The end result should be an OCR system that Cambodian schools, libraries, 
and businesses can actually use. Something that turns physical Khmer documents 
into searchable, editable digital text without requiring a PhD to operate.
That's the goal, Build something that solves real problems for real people.


\section{Research Questions}
\label{sec:questions}

We're trying to answer some pretty fundamental questions about Khmer OCR. Here's what we need to figure out:

\begin{enumerate}
    \item \textbf{Can we make CRAFT and TrOCR work well with Khmer script?} These models were built for English and other Latin languages. Khmer has stacked characters and weird spacing. Will these architectures even work, or do we need to change them completely?

    \item \textbf{How much synthetic data do we actually need?} Everyone talks about generating fake training images, but how many is enough? Can synthetic data really replace real photos of Khmer text? And what kind of augmentation tricks actually help versus just adding noise?

    \item \textbf{What's the minimum dataset size to get decent results?} We don't have Google's budget for data collection. So what's the smallest amount of real annotated data we need to train a working system? Is 1,000 images enough? or 10,000 images? or even more?

    \item \textbf{Can one system handle both Khmer and English effectively?} Mixed language documents are everywhere in Cambodia. But can a single OCR pipeline really switch between two completely different writing systems? Or do we need separate models that somehow work together?

    \item \textbf{What accuracy can we realistically achieve in real conditions?} Lab conditions with clean fonts are one thing. Real documents with blur, shadows, and weird angles are another. What's a realistic target for character error rate on actual Cambodian documents?

    \item \textbf{How do we make this system actually usable?} Building a model that works in Python notebooks is easy. Making something that regular people can use through an API? That's harder. What's the best way to deploy this so it actually helps people?
\end{enumerate}

These questions matter because answering them will tell us whether this whole approach is worth pursuing. And if it works, how to make it work better.

\section{Rationale of the Study}
\label{sec:rationale}
       This research is motivated by several compelling factors. First, there is an urgent need to digitize and preserve Cambodia's vast textual heritage, including historical documents, educational materials, and cultural artifacts. Without effective OCR technology for Khmer script, this digitization process remains labor-intensive and prone to errors.

Second, the current limitations of OCR systems for Khmer significantly hinder educational and academic initiatives in Cambodia. Many educational institutions struggle to convert physical textbooks and learning materials into digital formats, impacting accessibility and modernization efforts in education.

Third, the unique challenges posed by Khmer script—from character stacking to the absence of word boundaries—present an opportunity to advance the field of OCR technology as a whole. Solutions developed for Khmer may benefit other scripts with similar characteristics.

Finally, improving Khmer OCR technology aligns with broader digital transformation goals in Cambodia, supporting efforts to preserve cultural heritage while enabling more efficient information processing and accessibility in various sectors.

\section{Limitations and Scope}
\label{sec:limitations}

While this research aims to advance Khmer OCR technology significantly, it is important to acknowledge certain limitations and define the scope of the study:

\begin{enumerate}
    \item The research focuses specifically on printed Khmer text and English text and does not address handwritten text recognition, which presents additional challenges requiring separate investigation.
    
    \item The study primarily considers modern Khmer fonts and typography, with limited coverage of historical or decorative text styles.
    
    \item While the system aims to handle various document quality levels, extremely degraded or damaged documents may fall outside the scope of reliable recognition.
    
    \item The study focuses on optical character recognition and does not extend to higher-level natural language processing tasks such as semantic analysis or machine translation.
    
    \item Resource constraints may limit the size and diversity of the training dataset, though efforts will be made to ensure sufficient representation of common use cases.
\end{enumerate}

These limitations help maintain a focused research scope while acknowledging areas that may require future investigation.

\section{Structure of the Thesis}
\label{sec:structure}

This thesis is organized into the following chapters:

\begin{enumerate}
    \item \textbf{Introduction}: Presents the research background, objectives, research questions, rationale, and scope of the study.
    
    \item \textbf{Literature Review}: Reviews existing OCR technologies, challenges in Khmer script recognition, and relevant deep learning approaches.
    
    \item \textbf{Methodology}: Details the proposed approach, including dataset preparation, model architecture, and training procedures.
    
    \item \textbf{Implementation}: Describes the technical implementation, including preprocessing techniques, model modifications, and system integration.
    
    \item \textbf{Results and Analysis}: Presents experimental results, performance analysis, and comparative evaluation with existing solutions.
    
    \item \textbf{Conclusion}: Summarizes key findings, contributions, and suggests directions for future research.
\end{enumerate}
Each chapter builds upon the previous ones to present a comprehensive study of Khmer OCR development.


\end{document}
