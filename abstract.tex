\begin{adjustwidth}{0.2cm}{0.2cm}

    % Khmer Abstract
    \begin{center}
        {\khmerfont\fontsize{15pt}{25pt}\selectfont សង្ខេបសេចក្ដី \par}
    \end{center}
    \phantomsection
    \label{khmer-abstract}
    \vspace{0.5cm}
    \khmernormal
    \small
    ក្នុងសហគមន៍បច្ចេកវិទ្យាព័ត៌មានសម័យថ្មី ការចាប់យកអត្ថបទចេញពីរូបភាព  – [ OCR ]
    (Optical Character Recognition) ក្លាយជាបច្ចេកវិទ្យាសំខាន់មួយដែលត្រូវបានប្រើប្រាស់
    យ៉ាងទូលំទូលាយ សម្រាប់បំលែងឯកសារសរសេរ ឬរូបភាពអក្សរឱ្យទៅជាអត្ថបទ អេឡិចត្រូនិច (digital text)។ 
    ការអភិវឌ្ឍ OCR សម្រាប់ភាសាខ្មែរ តែងតែប្រឈមនឹងបញ្ហាជាច្រើន ដោយសារកង្វះនៃប្រភពទិន្នន័យ 
    និងឯកសារសម្រាប់ train AI model។ ដើម្បីដោះស្រាយបញ្ហានេះ យើងបានបង្កើតទិន្នន័យសិប្បនិម្មិត 
    (Synthetic Dataset) ដោយប្រើវិធីសាស្ត្របច្ចេកទេសកម្រិតខ្ពស់។\par
    
    ក្នុងដំណើរការបង្កើតទិន្នន័យសិប្បនិម្មិត (Synthetic Dataset) រួមមាន៖
    \begin{itemize}
        \item វិធីសាស្ដ្រក្នុងការប្រមូលអត្ថបទចេញពីអ៊ីនធឺណិត មានដូចខាងក្រោម (Scrape data) ៖
        \begin{itemize}
            \item ដំណាក់កាលទីមួយ៖ យើងបានប្រមូលអត្ថបទចេញពី khsearch.com, Chuon-Nath-Dictionary, Alpha-Word, Google-Word, និងចុងក្រោយគឺ Huggingface.com ។
            \item ដំណាក់កាលទីពីរ៖ យើងបានសម្អាត ទិន្នន័យទាំងអស់នោះ ឆ្លងកាត់ដំណើរការ ដូចជា លុបចោលតួរអក្សរណាដែលមិនសូវមាន វត្តមាននៅលើ រូបភាព ញឹកញាប់ និងបានលុបចោល តួរអក្សរណាដែល Fonts renders អត់ចេញ។
            \item ដំណាក់កាលទីបី៖ ដំណាក់កាលមួយនេះ យើងបានធ្វើការ កាត់ប្រយោគទាំងអស់នោះ ជាពាក្យៗ ដោយប្រើប្រាស់ library ឈ្មោះ khmer-nltk
            \item ដំណាក់កាលទីបួន៖ ចុងក្រោយ ក៏បានរៀបចំជា ប្រយោគដែល មានប្រវែង Random ពី ១ អក្សរ រហូតដល់ ១១០ អក្សរ ។
        \end{itemize}
        \item បង្កើតរូបភាពដោយអនុវត្តតាមលក្ខខណ្ឌខាងក្រោម ៖ 
        \begin{itemize}
            \item ផ្ទៃខាងក្រោយចៃដន្យ (Apply Different backgrounds)
            \item បំពាក់ពុម្ពអក្សរផ្សេងៗគ្នា (Apply Different fonts)
            \item Noise: \texttt{gaussian\_noise}, \texttt{salt\_pepper\_noise}, \texttt{speckle\_noise}, \texttt{blur}
            \item បង្វិលអក្សរបន្តិច (random rotation text)
            \item បញ្ចូល Margin Randomly (1, 5) pixels
        \end{itemize}
        \item សរុបមកយើងបានបង្កើត Data ជាង ៤ លាន records សម្រាប់ train OCR model
    \end{itemize}
    \par
    
    Architecture OCR ត្រូវបានបែងចែកជា ២ ផ្នែក៖ \textbf{Text Detection} និង \textbf{Text Recognition}:\par
    
    \textbf{Text Detection}: យើងប្រើម៉ូដែល \textbf{CRAFT} ដោយបានធ្វើការ Train ឡើងវិញដោយ បាន annotation ទៅលើ 
    លើរូបភាពប្រហែល ៥០០ images និងសរុបចំនួន bounding box ជាង ១០,០០០ boxes។\par
    
    \textbf{Text Recognition}: យើងប្រើ \textbf{TrOCR base model} ចេញពី Microsoft (មាននៅក្នុង Hugging Face) ហើយបាន
    fine-tune ទៅលើ dataset ខ្មែរសិប្បនិម្មិត (Synthetic Dataset) ដើម្បីបង្កើនសមត្ថភាពក្នុងការសម្គាល់អក្សរខ្មែរ។\par
    
    លទ្ធផលសិក្សាបានបង្ហាញថា OCR របស់ពួកយើងអាចសម្គាល់អត្ថបទចេញពីរូបភាព បានដោយភាពត្រឹមត្រូវលើសពី ៩០\%។ 
    ដូច្នេះ ការសិក្សានេះបង្ហាញអំពីសក្តានុពលនៃការបង្កើត dataset និងការប្រើម៉ូដែលជំនាន់ថ្មី ដើម្បីអភិវឌ្ឍន៍ OCR ភាសាខ្មែរឱ្យមានប្រសិទ្ធភាពកាន់តែខ្ពស់។
    វាមានសមត្ថភាព អាចចាប់យកអត្ថបទមិនត្រឹមតែពាក្យខ្លីៗ ប៉ុណ្ណោះទេ តែវាក៏អាចធ្វើការចាប់យក ដូចជា មួយតួអក្សរដោយមួយតួអក្សរ, ពាក្យដោយពាក្យ, ប្រយោគដោយប្រយោគ រហូតដល់ មួយប្រយោគវែង ១១០ តួអក្សរថែមទៀតផង ។ ហើយលើសពីនោះទៀត វាក៏អាចធ្វើការ កំណត់សម្គាល់ទៅលើ ពីរ ភាសាចម្បង ទាំងភាសាខ្មែរ និងភាសាអង់គ្លេស ។    
    \vspace{2cm}
    
    % English Abstract
    \begin{center}
        {\bfseries\LARGE Abstract \par}
    \end{center}
    \phantomsection
    \label{abstract}
    \vspace{0.5cm}
    \englishfont
    \large
    
    OCR technology continues to evolve rapidly, but Khmer text 
    recognition still presents challenges—mainly because of the 
    limited availability of high-quality training data. In this work, 
    we introduce an end-to-end OCR approach for both Khmer and English text, 
    tailored to work well even under low-resource conditions. Our pipeline combines 
    reliable text detection and advanced recognition techniques, designed to reflect real-world multilingual usage in Cambodia.

    We began by manually collecting a dataset of 1,000 real-world images, 
    carefully annotating 13,200 text-line bounding boxes in both Khmer and English. 
    This gave us a strong foundation to train our text detection model using CRAFT, 
    which excels at detecting irregular text regions. Training on our custom-annotated 
    data, the CRAFT model achieved strong performance, with 90\% recall, 89\% precision, 
    and an F1-score of 86.8\%. These results show that with focused annotation efforts, 
    high-quality detection is achievable even without large public datasets.
    
    To further improve the system’s performance and handle a wider variety of input, 
    we generated a synthetic training dataset using a text-to-image method. We 
    collected a large corpus of Khmer and English sentences from the web and used 
    this to create thousands of synthetic images simulating different fonts, layouts, 
    and noise patterns. This synthetic data expanded the diversity of our training 
    set and allowed the model to better generalize to unseen real-world samples.
    
    For the recognition stage, we used TrOCR—a transformer-based architecture built 
    for OCR tasks. However, the original TrOCR processor was not designed to support 
    Khmer text and could not correctly interpret the language. To address this, 
    we customized the processor so that it could properly tokenize and decode Khmer 
    characters. This modification was essential to ensure that the model could 
    understand and accurately transcribe Khmer script.
    
    After training on both real and synthetic data, our TrOCR-based recognition model 
    showed strong results on real-world test images: a character error rate (CER) 
    of 0.02 overall, with 0.04 for Khmer, 0.01 for English, and 0.06 for lines 
    containing both languages.
    
    One of the key motivations behind this work is the increasing use of English 
    in Cambodia. While Khmer remains the national language, English is now widely 
    used as a second language and often appears alongside Khmer in signs, documents, 
    and digital platforms. By building a system that supports both languages, 
    we aim to match the multilingual nature of everyday text in Cambodia.
    
    Finally, unlike most academic research that never reaches real-world usage, 
    we’ve made our OCR system publicly accessible through a production-ready API. 
    This allows others to test and integrate the model into their own applications, 
    bridging the gap between research and practical deployment. According to recent 
    studies, nearly 80\% of research models never get deployed or made accessible—our 
    goal was to break that trend. By turning our model into a usable, open API, 
    we ensure that the results of this work are not only measurable in papers 
    but also usable by the community.


    
    \end{adjustwidth}